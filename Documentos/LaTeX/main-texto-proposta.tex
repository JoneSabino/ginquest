%% Adaptado a partir de :
%%    abtex2-modelo-trabalho-academico.tex, v-1.9.2 laurocesar
%% para ser um modelo para os trabalhos no IFSP-SPO

\documentclass[
    % -- opções da classe memoir --
    12pt,               % tamanho da fonte
    openright,          % capítulos começam em pág ímpar (insere página vazia caso preciso)
    %twoside,            % para impressão em verso e anverso. Oposto a oneside
    oneside,
    a4paper,            % tamanho do papel. 
    % -- opções da classe abntex2 --
    %chapter=TITLE,     % títulos de capítulos convertidos em letras maiúsculas
    %section=TITLE,     % títulos de seções convertidos em letras maiúsculas
    %subsection=TITLE,  % títulos de subseções convertidos em letras maiúsculas
    %subsubsection=TITLE,% títulos de subsubseções convertidos em letras maiúsculas
    % Opções que não devem ser utilizadas na versão final do documento
    draft,              % para compilar mais rápido, remover na versão final
    MODELO,             % indica que é um documento modelo então precisa dos geradores de texto
    TODO,               % indica que deve apresentar lista de pendencias 
    % -- opções do pacote babel --
    english,            % idioma adicional para hifenização
    brazil              % o último idioma é o principal do documento
    ]{ifsp-spo-inf-ctds}

        
% ---

% --- 
% CONFIGURAÇÕES DE PACOTES
% --- 
%\usepackage{etoolbox}
%\patchcmd{\thebibliography}{\chapter*}{\section*}{}{}


% ---
% CAPA e FOLHA DE ROSTO
% ---
\titulo{GinQuest: Aplicativo para criação e gerenciamento de gincanas}

\renewcommand{\imprimirautor}{
\begin{tabular}{lr}
Jones Sabino Silva & SP1672576 \\
Murilo Vicente da Silva & SP1674706 \\
Renata Monteiro Gadelha & SP1666339 \\
Rodrigo Bressan de Souza & SP167031X \\
Victor Hiroshi Castro Kawamoto & SP1670425 \\
\end{tabular}
}

\tipotrabalho{Projeto da Disciplina de Prática e Gerenciamento de Projetos}

\disciplina{A6PGP - Prática e Gerenciamento de Projetos}

\preambulo{Proposta de projeto para disciplina de Prática e Gerenciamento de Projetos}

\data{2019}

% Utilizar o Nome Completo, abntex tem orientador e coorientador
% então vão ser utilizados na definição de professor
\renewcommand{\orientadorname}{Professor:}
\orientador{José Braz de Araújo}
\renewcommand{\coorientadorname}{Professor:}
\coorientador{Ivan Francolin Martinez}

% ---

% Configurações de aparência do PDF final


% informações do PDF
\makeatletter
\hypersetup{
        %pagebackref=true,
        pdftitle={\@title}, 
        pdfauthor={\@author},
        pdfsubject={\imprimirpreambulo},
        pdfcreator={LaTeX with abnTeX2},
        pdfkeywords={abnt}{latex}{abntex}{abntex2}{trabalho acadêmico}, 
        colorlinks=true,            % false: boxed links; true: colored links
        linkcolor=blue,             % color of internal links
        citecolor=blue,             % color of links to bibliography
        filecolor=magenta,              % color of file links
        urlcolor=blue,
        bookmarksdepth=4
}
\makeatother
% --- 

% ---

% ----
% Início do documento
% ----
\begin{document}

% Retira espaço extra obsoleto entre as frases.
\frenchspacing 

\pretextual

% ---
% Capa - Para proposta a folha de rosto é suficiente pois é mais completa.
% ---
\imprimirfolhaderosto
% ---

% ----------------------------------------------------------
% ELEMENTOS TEXTUAIS
% ----------------------------------------------------------
\textual

% \chapter{Introdução}
\section{Introdução}
Segundo \citeauthor{analaura:2016} o marketing de relacionamento tem se destacado como ponto importante de manutenção e captação de clientes. As mídias sociais despontam como uma oportunidade para a interação entre as pessoas e, significativamente, como meio de relacionamento entre
empresa e consumidor.

O alto nível de relacionamento entre a empresa e o consumidor pode contribuir para o sucesso do negócio, ao transformar o consumidor em fã, reduzir os
investimentos em publicidade e levar o
consumidor a indicar os produtos (\citeauthor{guilherme:2013}, 2013)

Uma das mais importantes definições
de engajamento do cliente foi elaborada por \citeauthor{jenny:2010} afirmando que é a “manifestação de comportamento dos clientes em direção a uma marca ou empresa, que vai além da compra, resultando em condutores motivacionais”.

\section{Proposta}
% \section{Plataforma My Place}
O presente projeto tem o objetivo de desenvolver um aplicativo para dispositivos móveis que funciona como uma plataforma de criação e gerenciamento de gincanas onde empresas possam cadastrar diversas atividades disponíveis para serem realizadas pelos usuários.

Além do usuário final se divertir com as atividades o mesmo ainda seria bonificado com prêmios e descontos. As empresas ganharão com o engajamento do usuário com a sua marca além de publicações relacionadas à gincana sendo realizada em outras redes sociais.

O foco do aplicativo são as empresas, mas os usuários também podem criar gincanas privadas a serem compartilhadas com os amigos.

Todas as gincanas possuem raking de pontuação (que levará também em conta o tempo de realização da atividade e a complexidade da tarefa) e seu administrador pode definir uma premição na criação da gincana: brinde, número da sorte, desconto ou trofeu virtual.

O administrador além de criar as gincanas, denifir as atividades e verificar o andamento da mesma, também pode cadastrar moderadores para auxiliar no gerenciamento. 

Para um administrador ser reconhecido como representante oficial de uma empresa é necessário ser aprovado em um ou mais critérios de validação (depende da empresa), como cadastrar uma gincana com premiação e inserir o número de cadastro da promoção na Caixa Econômica Federal, inserir código de validação gerado pelo aplicativo no site oficial da empresa, verificar nome no registro da empresa, entre outras opções.

O aplicativo também conta com conta de usuários certificados, que poderão ser indicados para serem moderadores de gincanas. Moderadores não podem participar das gincanas durante o periodo que estão moderando qualquer gincana.
 
\section{Lista de cadastro de atividades}
\begin{itemize}
\item Escolha de um local no mapa com possibilidade de limitar os dias e horários. Ex: Av. Paulista no dia 01/03/2019 à meia noite;
\item Leitura de QR Code. Ex: Encontre o QR Code no primeiro piso do shopping D;
\item Realizar publicação no facebook, twitter ou instagram de um conteúdo pré-programado. Ex: Aplicativo realizar uma publicação informando que o usuário está participando de uma gincana;
\item Perguntas com resposta já definidas com a possibilidade de limitar o tempo de resposta. Ex: Qual era o nome do fundador da Apple? Respostas A, B, C, D e tempo limite de 5 segundos;
\item Fotos, vídeos e publicações realizando atividades específicas (precisaria de um moderador caso o administrador desejasse realmente validar);
\item Atividade específica a ser validada por um moderador. Ex: Dançar macarena na loja da marca e o funcionário validar a atividade (ele poderia simplesmente ter uma QRCode validadora na loja);
\item Criar uma ordem específica para as atividades serem cumpridas. Ex: Primeiro realizar a atividade da Paulista e depois fazer a dança da macarena.
\end{itemize}

\section{Gerenciamento}
Programas utilizados para auxiliar no gerenciamento do projeto:

\begin{itemize}
\item Whatsapp: Ferramenta principal para comunicação;
\item Messenger: Ferramenta de apoio à comunicação quando a ferramenta anterior não puder ser utilizada;
\item Google Drive: Ferramenta para armazenamento compartilhado do material de apoio (material de referência, rascunhos, materiais gráficos, etc): 
\item Overleaf: Ferramenta para escrita e formatação compartilhada do relatório final;
\item Subversion (Tortoise SVN): Ferramenta cliente para o sistema de controle de versão SVN (Repositório);
\item Monday: Ferramenta para o gerenciamento do projeto.
\end{itemize}

\section{Desenvolvimento}
Tecnologias e ferramentas utilizadas no desenvolvimento do projeto:

\begin{itemize}
\item Banco de Dados: Relacional a ser definido;
\item Linguagem de Programação Back-end: Javascript utilizando Node.js;
\item Linguagem de Programação Front-end Mobile: Javascript utilizando o framework React Native;
\item Ambiente de implantaçao Plataforma na Nuvem a ser definida.
\end{itemize}

% ----------------------------------------------------------
% Referências bibliográficas
% ----------------------------------------------------------
\bibliography{referencias,exemplos/abntex2-doc-abnt-6023}

\end{document}