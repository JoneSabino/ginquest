
% ----------------------------------------------------------
% Introdução (exemplo de capítulo sem numeração, mas presente no Sumário)
% ----------------------------------------------------------
\chapter[Introdução]{Introdução}

O ser humano naturalmente possui intrinsecamente o instinto competitivo, desde seus primórdios lutando pela sua sobrevivência na natureza. Entretanto, essa competitividade pode se tornar descontrolada dependendo de como o indivíduo é apresentado aos desafios que aparecem em sua rotina.

Há duas formas de mostrar isto, a primeira forma os apresenta (geralmente quando ainda crianças) como sendo situações onde somente a vitória importa e os únicos que merecem reconhecimento e crédito são os ganhadores, independente dos esforços dos concorrentes. Com esta visão, aqueles que não obtiveram resultados tão bons são vistos como perdedores e mais fracos, e esse estímulo negativo pode tornar a pessoa extremamente competitiva e futuramente pode sofrer com problemas como: ansiedade, estresse e nervosismo \citeonline{harris} . Já a segunda forma mostra como sendo algo rotineiro, em que nem sempre será possível ganhar e ser visto como o melhor de todos, demonstrando assim, que todos merecem reconhecimento, principalmente aqueles que enfrentaram dificuldades em suas jornadas e persistiram em concluir.

Por isso, um dos meio de habituar, principalmente as crianças, a conviver com desafios é por meio de gincanas, pois são competições geralmente coletivas e demonstram que cada um possui suas forças e fraquezas, e que isso torna os seres humanos iguais. Além disso, muitas das gincanas incentivam o trabalho em equipe, estimulando a sociabilização e o pensamento coletivo \citeonline{unilever}.

É comum as pessoas se depararem com a depreciação dos adultos para com os jogos de forma geral, pois os mesmos veem como algo inútil ou infantil, porém há motivos para que isso aconteça e motivos também que comprovam que esta visão está equivocada, pois primeiramente para comprovar a depreciação é verificável que os adultos estão diretamente ligados a conquistas de bens materiais, e por conta dos jogos não necessariamente possuírem essa finalidade acabam sendo classificados como perda de tempo. Por outro lado, ao analisar a história da humanidade é possível identificar que os jogos foram considerados sagrados por séculos, por muitas civilizações, como os egípcios, gregos e romanos, por exemplo, e para estes povos era considerado uma honra e um privilégio poder participar (SANTOS, 2015), logo não é correto menosprezar os momentos de diversão e competição que os jogos propõe porque tem uma importância social e cultural relevante.


\section{Questão de Pesquisa}
\lipsum[1]

\section{Objetivos}
\lipsum[1]

\subsection{Objetivo Principal}
\lipsum[1]

\subsection{Objetivos Secundários}
\lipsum[1]

\section{Justificativa}
\lipsum[1]

\section{Estrutura do Estudo}
\lipsum[1]

