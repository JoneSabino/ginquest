%% Adaptado a partir de :
%%    abtex2-modelo-trabalho-academico.tex, v-1.9.2 laurocesar
%% para ser um modelo para os trabalhos no IFSP-SPO

\documentclass[
    % -- opções da classe memoir --
    12pt,               % tamanho da fonte
    openright,          % capítulos começam em pág ímpar (insere página vazia caso preciso)
    twoside,            % para impressão em verso e anverso. Oposto a oneside
    %oneside,
    a4paper,            % tamanho do papel. 
    % -- opções da classe abntex2 --
    %chapter=TITLE,     % títulos de capítulos convertidos em letras maiúsculas
    %section=TITLE,     % títulos de seções convertidos em letras maiúsculas
    %subsection=TITLE,  % títulos de subseções convertidos em letras maiúsculas
    %subsubsection=TITLE,% títulos de subsubseções convertidos em letras maiúsculas
    % Opções que não devem ser utilizadas na versão final do documento
    openany             %capitulos podem começar em página pares também
    draft,              % para compilar mais rápido, remover na versão final
    MODELO,             % indica que é um documento modelo então precisa dos geradores de texto
    TODO,               % indica que deve apresentar lista de pendencias 
    % -- opções do pacote babel --
    english,            % idioma adicional para hifenização
    brazil              % o último idioma é o principal do documento
    ]{ifsp-spo-inf-ctds}

        
% ---

% --- 
% CONFIGURAÇÕES DE PACOTES
% --- 
%\usepackage{etoolbox}
%\patchcmd{\thebibliography}{\chapter*}{\section*}{}{}


% ---
% CAPA e FOLHA DE ROSTO
% ---
\titulo{GinQuest: Aplicativo para criação e gerenciamento de gincanas}

\renewcommand{\imprimirautor}{
\begin{tabular}{lr}
Jones Sabino Silva & SP1672576 \\
Murilo Vicente da Silva & SP1674706 \\
Renata Monteiro Gadelha & SP1666339 \\
Rodrigo Bressan de Souza & SP167031X \\
Victor Hiroshi Castro Kawamoto & SP1670425 \\
\end{tabular}
}

\tipotrabalho{Projeto da Disciplina de Prática e Gerenciamento de Projetos}

\disciplina{A6PGP - Prática e Gerenciamento de Projetos}

\preambulo{Prova de conceito do projeto para disciplina de Prática e Gerenciamento de Projetos}

\data{2019}

% Utilizar o Nome Completo, abntex tem orientador e coorientador
% então vão ser utilizados na definição de professor
\renewcommand{\orientadorname}{Professor:}
\orientador{José Braz de Araújo}
\renewcommand{\coorientadorname}{Professor:}
\coorientador{Ivan Francolin Martinez}

% ---

% Configurações de aparência do PDF final


% informações do PDF
\makeatletter
\hypersetup{
        %pagebackref=true,
        pdftitle={\@title}, 
        pdfauthor={\@author},
        pdfsubject={\imprimirpreambulo},
        pdfcreator={LaTeX with abnTeX2},
        pdfkeywords={abnt}{latex}{abntex}{abntex2}{trabalho acadêmico}, 
        colorlinks=true,            % false: boxed links; true: colored links
        linkcolor=blue,             % color of internal links
        citecolor=blue,             % color of links to bibliography
        filecolor=magenta,              % color of file links
        urlcolor=blue,
        bookmarksdepth=4
}
\makeatother 
% --- 

% ---

% ----
% Início do documento
% ----
\begin{document}

% Retira espaço extra obsoleto entre as frases.
\frenchspacing 

\pretextual

% ---
% Capa - Para proposta a folha de rosto é suficiente pois é mais completa.
% ---
\imprimirfolhaderosto
% ---

% ----------------------------------------------------------
% ELEMENTOS TEXTUAIS
% ----------------------------------------------------------
\textual

Este documento visa comprovar a aderência das atividades desenvolvidas  pela equipe GinQuest até a data da apresentação da prova de conceito, além de identificar as tecnologias utilizadas no seu desenvolvimento, e  representar sua arquitetura.

O desenvolvimento do projeto foi dividido em três frentes: infraestrutura, aplicativo e API. A arquitetura geral do sistema foi apresentada na FIgura 1.

\vspace{1cm}

\begingroup
\let\clearpage\relax
\chapter{Infraestutura}
\endgroup

A infraestrutura do projeto é composta por um servidor WEB e um servidor de banco de dados, ambos utilizando serviços de computação em nuvem fornecidos pelo Google Cloud Platform, o GCP. A arquitetura de infraestrutura do projeto foi representada na Figura 2.


\begin{itemize}
\item AWS Elastic Compute Cloud - EC2: Serviço web da Amazon que disponibiliza capacidade computacional redimensionável e segura. Esse serviço hospeda a instância do servidor web t2.micro com o sistema operacional Windows Server 2016;
\item WS Relational Database Service - RDS: Serviço de banco de dados relacional da Amazon, que permite a configuração, operação e escalabilidade do banco de dados. Esse serviço hospeda a instância do servidor de banco de dados PostgreSQL.
\item WS Virtual Private Cloud - VPC: Rede virtual que permite provisionar uma seção da nuvem AWS logicamente isolada, e permite gerenciar recursos da Amazon Web Services. Esse serviço foi utilizado para permitir e gerenciar a ativação de recursos da AWS em uma rede virtual dedicada a conta pela equipe.
\item AWS Route 53: ...;
\item PostgreSQL: ...;
\item Win-acme: ...;
\item IISCrypto: ...;
\item Windows Server 2016: ...;
\item IIS10: ...
\end{itemize}

\begingroup
\let\clearpage\relax
\chapter{Aplicativo}
\endgroup

O aplicativo será desenvolvido utilizando a tecnologia IONIC e se comunicarão com APIs desenvolvidas pela equipe do projeto, assim como APIs de terceiros (Google Maps API e Facebook API), como explicitado na Figura 3.

\begin{itemize}
\item Ionic: ...;
\item LAngula....
\end{itemize}

\begingroup
\let\clearpage\relax
\chapter{API}
\endgroup

A frente API visa o desenvolvimento de Web APIs que serão consumidas pelos aplicativos móveis. Essas APIs são responsáveis por acessar o banco de dados e enviar e receber informações para o aplicativo.

\begin{itemize}
\item NET Framework: ...;
\item Entity Framework: ....
\item Sonarqube: ....
\end{itemize}


% ----------------------------------------------------------
% Referências bibliográficas
% ----------------------------------------------------------
\bibliography{referencias,exemplos/abntex2-doc-abnt-6023}

\end{document}