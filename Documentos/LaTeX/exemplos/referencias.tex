\subsection{Citações / Referências}

Existem diversas formas de citação observe os exemplos :

\begin{itemize}
    \item \cite{UML:JACOBSON} | \cite{POWELL:2006} \\ 
        \cite{SCRUMGUIDE:2013} | \cite{urani1994}

    \item \citeonline{UML:JACOBSON} | \citeonline{POWELL:2006} \\
        \citeonline{SCRUMGUIDE:2013} | \citeonline{urani1994}

    \item \citeauthoronline{UML:JACOBSON}| \citeauthoronline{POWELL:2006} \\
        \citeauthoronline{SCRUMGUIDE:2013} | \citeauthoronline{urani1994}

    \item \citeauthor{UML:JACOBSON}| \citeauthor{POWELL:2006} \\
        \citeauthor{SCRUMGUIDE:2013}| \citeauthor{urani1994}
        
    \item \url{http://mirrors.ibiblio.org/CTAN/macros/latex/contrib/abntex2/doc/abntex2cite-alf.pdf}
\end{itemize}

Os dados devem ser definidos corretamente nos arquivos \textquote{.BIB} para a correta formatação no texto e na lista de referências.

Palavras que devem ser apresentadas no glossário devem ser citadas especificamente no texto utilizando os comandos de glossário como : \gls{tag}


Abreviaturas podem ser referenciadas diretamente 
na versão reduzida \textquote{\acs{ifsp}} \space  
ou longa \textquote{\acl{ifsp}}

\begin{itemize}
    \item Autor com diversas publicações no mesmo ano : \url{https://github.com/abntex/biblatex-abnt/issues/20}
\end{itemize}