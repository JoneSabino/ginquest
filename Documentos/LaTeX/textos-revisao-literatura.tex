% ---
% Capitulo de revisão de literatura
% ---
\chapter{Revisão da Literatura}
O presente capítulo é responsável pela revisão de literatura, tendo como objetivo tornar os temas abordados do projeto GinQuest de mais fácil compreensão para que sejam mitigadas possíveis dúvidas sobre os assuntos tratados.\
% ---

% ---
\section{Gincanas}
Gincanas são atividades constituídas por desafios que podem ser referentes a esportes, perguntas e respostas, envolvendo cultura e arte, e também pode ser baseado em achar algum artefato escondido, baseado na descobertas de pistas para sua conclusão.

Atividades que estimulam a competição de forma saudável são excelentes como diversão e ajudam indivíduos a terem menos problemas com ansiedade e nervosismo, por exemplo, pois naturalmente estes tipos de emoções acabam sendo controladas mediante a exposição às atividades propostas em gincanas, e comumente quem possui bastante controle emocional costuma ter melhores performances que outras pessoas, incentivando assim, os participantes a aprimorar essas habilidades para se manterem competitivos.

\section{Marketing de Relacionamento}
É uma estratégia utilizada pelas empresas para que os clientes tenham uma proximidade maior e sejam fiéis às marcas. Esta forma de marketing possui como objetivo fazer com que os consumidores sintam-se engajados a acreditar nas soluções propostas que determinada empresa propõe, e ainda também em casos onde há sucesso na estratégia: o cliente recomenda os serviços e produtos oferecidos pela empresa para outros indivíduos e defende a marca contra comentários negativos como críticas, chacotas e ofensas \citeonline{thome}.

O marketing de relacionamento tem se tornado comum, principalmente em grandes empresas, pois não basta apenas oferecer um bom produto, é necessário satisfazer o desejo do cliente e o agradar para que haja motivação que o mesmo compre novamente, ou ainda melhor, divulgue para seus conhecidos a boa experiência que teve com a empresa na qual teve interação de compra.

Este tem sido um investimento muito assertivo para as marcas, pois um novo cliente pode custar sete vezes mais do que fidelizar aquele que já compra normalmente, conforme \citeonline{rocha}.

\section{Assunto 3}
\lipsum[3-5]
\section{Assunto X}
\lipsum[2-4]
% ---
